\documentclass[journal,10pt,twocolumn]{article}
\usepackage{graphicx}
\usepackage{caption} 
\usepackage{hyperref}
\usepackage[margin=0.5in]{geometry}
\usepackage{booktabs}
\usepackage{array}
\usepackage{amsmath}   % for having text in math mode
\usepackage{mathtools}
\usepackage{enumitem}
\usepackage{atbegshi}% http://ctan.org/pkg/atbegshi
\AtBeginDocument{\AtBeginShipoutNext{\AtBeginShipoutDiscard}}
\newcommand{\myvec}[1]{\ensuremath{\begin{pmatrix}#1\end{pmatrix}}}
\let\vec\mathbf
\newcommand{\mydet}[1]{\ensuremath{\begin{vmatrix}#1\end{vmatrix}}}
\providecommand{\brak}[1]{\ensuremath{\left(#1\right)}}
\newcommand{\solution}{\noindent \textbf{Solution: }}
\let\vec\mathbf
\begin{document}
\begin{center}
\title{\textbf{  Unit Vector Perpendicular}}
\date{\vspace{-5ex}} %Not to print date automatically
\maketitle
\end{center}
\setcounter{page}{1}
\section{12$^{th}$ Maths - Chapter 10}
\textbf{This is Problem-2 from Exercise 4.2}
\begin{enumerate}

\item Find a unit vector  perpendicular to each of a vector $\bar{a}+\bar{b} \text{ and }\bar{a}-\bar{b}$ where  $\overrightarrow{a}=3\hat{i}+2\hat{j}+2\hat{k}\text{ and }\overrightarrow{b}=\hat{i}+2\hat{j}-2\hat{k}$
\section{Solution}
A unit vector  perpendicular
\begin{align} 
(\vec{a}+\vec{b})^\top\vec{x}&=0 \\
(\vec{a}-\vec{b})^\top\vec{x}&=0\\
\myvec{(\vec{a}+\vec{b})^\top\\ (\vec{a}-\vec{b})^\top}\vec{x}&=0\\
\myvec{\vec{a}+\vec{b}& \vec{a}-\vec{b}}^\top\vec{x}&=0\label{4}
\end{align}
Here.

\begin{align}
\vec{a}=\myvec{3\\2\\2},
\vec{b}=\myvec{1\\2\\-2}
\end{align}

\begin{align}
\vec{a+b}=\myvec{3\\2\\2}+\myvec{1\\2\\-2}=\myvec{4\\4\\0}\label{7}\\
\vec{a-b}=\myvec{3\\2\\2}-\myvec{1\\2\\-2}=\myvec{2\\0\\4}\label{8}
\end{align}
Now using the formula substituing (7) and (8) in (4) and equating,
\begin{align}
\myvec{\vec{a}+\vec{b}& \vec{a}-\vec{b}}^\top\vec{x}&=0\\
\myvec{
4&2\\
4&0\\
0&4
}^\top \vec{x}&=0\\
\myvec{
4&4&0\\
2&0&4
}
\xleftrightarrow[]{R_1=\frac{R_1}{4}}
\myvec{
1&1&0\\
2&0&4
}\vec{x}&=0\\\xleftrightarrow[]{R_2=\frac{R_2}{2}}
\myvec{
1&1&0\\
1&0&2
}\vec{x}&=0\\
\xleftrightarrow[]{R_2={R_1}-{R_2}}
\myvec{
1&1&0\\
0&-1&2
}\vec{x}&=0\\
\xleftrightarrow[]{R_2=\frac{R_2}{-1}}
\myvec{
1&1&0\\
0&1&-2
}\vec{x}&=0\\
\xleftrightarrow[]{R_1={R_1}-{R_2}}
\myvec{
1&0&2\\
0&1&-2
}\vec{x}&=0
\end{align}
From (15), we get two equations which is $x_1+2x_3=0$ and $x_2-2x_3=0$
\begin{align}
\vec{x}_1+2\vec{x}_3&=0\\
\vec{x}_1-2\vec{x}_3&=0\\
x_1&=\myvec{-2x_3\\2x_3\\x_3}\\
&=x_3\myvec{-2\\2\\1}
\end{align}
A unit vector perpendicular to 
\begin{align}
\hat{\vec{x_1}} &=\frac{1}{3}\myvec{-2\\2\\1}\vec{x_3}
\end{align}
\end{enumerate} 
\end{document}